\section{Yoneda reduction, Kan extensions}\label{sec:tre}
% \epigraph{The reason why this technique is so fast is that you are not trying to cut them; you are throwing your sword into them.}{M\@. Hatsumi}
% \subsection{The ninja Yoneda lemma}
One of the most famous results about the category $\Cat(\C^\text{op}, \Sets)$ of presheaves on a category $\C$ is that every object in it can be canonically presented as a colimit of representable functors; see \cite[Theorem \textbf{III.7.1}]{McL} for a description of this classical result (and its dual holding in $\Cat(\C, \Sets)$).

Now, co/end calculus allows us to rephrase this result in an extremely compact way, called \emph{Yoneda reduction}; in a few words, it says that \emph{every co/presheaf can be expressed as a co/end}\footnote{The reader looking for a nifty explanation of this result should wait for a more thorough discussion, which can be deduced from the material in \S\refbf{section:weight}, thanks to the machinery of \emph{weighted co/limits}.}.
\begin{proposition}[ninja Yoneda Lemma]\label{ninjayo}
For every functor $K\colon \C^\text{op}\to\cate{Sets}$ and $H\colon \C\to \cate{Sets}$, we have the following isomorphisms (natural equivalences of functors):
\begin{gather*}
(i)\; K\cong \int^cKc\times \C(\firstblank,c) \qquad (ii)\; K\cong \int_c Kc^{\C(c,\firstblank)}\\
(iii)\; H\cong \int^cHc\times \C(c,\firstblank) \qquad (iv)\; H\cong \int_cHc^{\C(\firstblank,c)}\\
\end{gather*}
\end{proposition}
\begin{remark}
The name \emph{ninja Yoneda lemma} is a pun coming from a \texttt{Math-Overflow} \href{http://mathoverflow.net/a/20451}{comment} by T\@. Leinster, whose content is basically the proof of the above statement:
\begin{quote}\small 
Th[e above one is] often called the \emph{Density Formula}, [\dots] or (by Australian ninja category theorists) simply the Yoneda Lemma. (but Australian ninja category theorists call \emph{everything} the Yoneda Lemma\dots).
\end{quote}
\end{remark}
Undoubtedly, there is a link between the above result and the Yoneda Lemma we all know: in fact, the proof heavily relies on the Yoneda isomorphism, and in enriched setting (see \cite[\S\textbf{I.5}]{dubuc1970kan}) the ninja Yoneda lemma, interpreted as a theorem about Kan extensions, is \emph{equivalent} to the one from the Northern hemisphere. 

We must admit to feel somewhat unqualified to properly discuss the topic, as we live in the wrong hemisphere of the planet to claim any authority on it, and to be acquainted with the rather unique taste of Australian practicioners in choosing evocative (or obscure) terminology. Nevertheless, along the whole note, we keep the name ``ninja Yoneda Lemma'' as a (somewhat witty) nickname for the above isomorphisms, without pretending any authoritativeness whatsoever.
\begin{proof}
We prove case $(i)$ only, all the others being totally analogous. We put a certain emphasis on the style of this proof, as it is paradigmatic of most of the subsequent ones. Consider the chain of isomorphisms
\begin{align*}
\cate{Sets}\Big(\displaystyle \int^{c\in\C} Kc\times \C(x,c),y\Big)&\cong \int_{c\in\C} \cate{Sets}\big( Kc\times \C(x,c),y \big)\\
&\cong \int_{c\in\C}\cate{Sets}(\C(x,c),\cate{Sets}(Kc,y))\\
&\cong \Nat(\C(x,\firstblank),\cate{Sets}(K-,y))\\
&\cong\cate{Sets}(Kx,y)
\end{align*}
where the first step is motivated by the coend-preservation property of the $\hom$ functor, the second follows from the fact that $\Sets$ is a cartesian closed category, where 
\[
\Sets(X\times Y, Z)\cong \Sets(X, \Sets(Y,Z))
\]
for all three sets $X,Y,Z$ (naturally in all arguments), and the final step exploits Theorem \refbf{naturalu} plus the classical Yoneda Lemma.

Every step of this chain of isomorphisms is natural in $y$; now we have only to notice that the natural isomorphism of functors
\[
\Sets\Big(\textstyle \int^c Kc\times \C(x,c),y\Big)\cong \Sets(Kx,y)
\]
ensures that there exists a (natural) isomorphism $\int^c Kc\times \C(x,c)\cong Kx$. This concludes the proof and it is, if you want, a way to motivate and partially solve Exercise \textbf{1}.\refbf{ex1:jointlypreserve}.
\end{proof}
From now one we will make frequent use of the notion of ($\Sets$-)\emph{tensor} and ($\Sets$-)\emph{cotensor} in a category; these standard definitions are in the chapter of any book about enriched category theory (see for example \cite[Ch. \textbf{6}]{Bor2}, its references, and in particular its Definition \textbf{6.5.1}, which we report for the ease of the reader:
\begin{definition}[Tensor and cotensor in a $\V$-category]\label{tenscotens}
In any $\V$-enriched category $\C$ (see \cite[Def\@. \textbf{6.2.1}]{Bor2}), the \emph{tensor} $\cdot\colon \V\times \C\to \C$ is a functor $(V, c)\mapsto V\cdot c$ such that there is the isomorphism \[ \label{ten} \C(V\cdot c, c')\cong \V(V, \C(c,c')),\] natural in all components; dually, the \emph{cotensor} in an enriched category $\C$ is a functor $(V, c)\mapsto c^V$ (contravariant in $V$) such that there is the isomorphism \[ \label{coten} \C(c', c^V)\cong \V(V, \C(c',c)),\] natural in all components. 
\end{definition}
\begin{example}
Every co/complete, locally small category $\C$ is naturally $\Sets$-co/tensored by choosing $c^V\cong \prod_{v\in V}c$ and $V\cdot c\cong \coprod_{v\in V}c$.
\end{example}
\begin{remark}\label{cocotens}
The tensor, hom and cotensor functors are the prototype of a \textsc{thc} \emph{situation} (see our Remark \refbf{tiaccaci} and \cite[\S \textbf{1.1}]{Gray1980} for a definition); given the $\hom$-objects of a $\V$-category $\C$, the tensor $\cdot\colon \V\times \C\to \C$ and the cotensor $(\secondblank)^{(\firstblank)}\colon \V^\text{op}\times \C\to \C$ can be characterized as adjoint functors: usual co/continuity properties of the co/tensor functors are implicitly derived from this characterization.
\end{remark}
\begin{remark}[\itsnonsense, The Yoneda embedding is a Dirac delta]\label{dirac}
In functional analysis, the Dirac delta appears in the following convenient abuse of notation:
\[
\int_{-\infty}^\infty f(x) \delta(x-y)dx = f(y)
\]
(the integral sign, here, is not a co/end). Here $\delta_y(x)  \coloneqq \delta(x-y)$ is the \emph{$y$-centered delta-distribution}, and $f\colon \mathbb{R}\to \mathbb{R}$ is a continuous, compactly supported function on~$\mathbb{R}$.

It is really tempting to draw a parallel between this relation and the ninja Yoneda lemma, conveying the intuition that representable functors on an object $c\in\C$ play the r\^ole of $c$-centered delta-distributions. If the relation above is written as $\langle f,\delta_y\rangle = f(y)$, interpreting integration as an inner product between functions, then the ninja Yoneda lemma says formally the same thing: for each presheaf $F\colon \C^\text{op} \to \Sets$, the ``inner product'' $\langle \yon_c, F\rangle = \int^x \yon_c(x)\times Fc$ equals $Fc$ (obviously, there's nothing special about sets here).
\end{remark}
\subsection{Kan extensions as co/ends}
\begin{definition}\label{kann}
Given a functor $F\colon \C\to \D$, its \emph{left} and \emph{right Kan extensions} are defined\footnote{This is not true, strictly speaking. Nevertheless we prefer to cheat the reader with this useful insight instead of obscuring the general idea keeping track of all possible pathologies. Furthermore, we are interested only in the cases when Kan extensions can be written as co/ends, so we will not consider any pathology whatsoever and we can safely assume that this is a proper definition.} to be, respectively, the left and right adjoint to the ``precomposition'' functor
\[
F^*\colon \Cat(\D,\cate E)\to \Cat(\C,\cate E)
\]
given by $H\mapsto F^*(H)=H\circ F$, in such a way that there are two  
isomorphisms
\begin{gather*}
\Nat(\Lan_FG,H)\cong \Nat(G, H\circ F)\\
\Nat(H\circ F,G)\cong \Nat(F, \Ran_FG).
\end{gather*}
\end{definition}
Now, we want to show that in ``nice'' situations it is possible to describe Kan extensions via co/ends: whenever the co/tensors (see Def. \refbf{tenscotens}) involved in the definition of the following co/ends exist in $\D$ for any choice of functors $F,G$ (a blatant example is when $\D$ is co/complete, since in that case as said before it is always $\Sets$-co/tensored by suitable iterated co/products, see the remark above \refbf{cocotens}), then the left/right Kan extensions of $G\colon \C\to\cate{E}$ along $F\colon \C\to \D$ exists and there are isomorphisms
\[\label{kanend}
\Lan_FG\cong \int^c \D(Fc,\firstblank)\cdot Gc\qquad\qquad 
\Ran_FG\cong \int_c Gc^{\D(\firstblank,Fc)}.
\]
\begin{proof}
The proof consists of a string of canonical isomorphisms, exploiting simple remarks in elementary Category Theory and the results established so far: the same argument is offered in \cite[Thm\@. \textbf{X.4.1, 2}]{McL}.
\begin{align*}
\Nat\Big( \int^c \D(Fc,\firstblank)\cdot Gc,H\Big)&\overset{\refbf{naturalu}}{\cong} \int_x\D\Big( \int^c \D(Fc,x)\cdot Gc,Hx \Big)\\
&\overset{\refbf{commuhom}}{\cong} \int_{cx}\D(\D(Fc,x)\cdot Gc,Hx)\\
&\overset{(\refbf{ten})}{\cong} \int_{cx}\cate{Sets}(\D(Fc,x),\cate{E}(Gc,Hx))\\
&\overset{\refbf{naturalu}}{\cong}\int_c\Nat(\D(Fc,\firstblank),\cate{E}(Gc,H-))\\
&\overset{\text{Yon}}{\cong} \int_c \cate{E}(Gc,HFc)\cong\Nat(G,HF).
\end{align*}
The case of $\Ran_FG$ is dually analogous.
\end{proof}
\begin{remark}
This is the pattern of every ``proof by coend-juggling'' we will meet in the rest of the paper; from now on we feel free to abandon a certain pedantry in justifying every single deduction in the chains of isomorphisms leading to conclude a proof.
\end{remark}
\begin{proposition}
Left/right adjoint functors commute with left/right Kan extensions, whenever they can be expressed as the coends above.
\end{proposition}
\begin{proof}
An immediate corollary of Theorem \refbf{coconti}, once it has been proved that a left adjoint commutes with tensors, \ie $F(X\cdot a)\cong X\cdot Fa$ for any $(X,a)\in \Sets\times \C$.
\end{proof}
\begin{example}
Let $T\colon \C\to \C$ be a \emph{monad} on the category $\C$; the \emph{Kleisli category} $\cate{Kl}(T)$ of $T$ is defined having the same objects of $\C$ and morphisms $\cate{Kl}(T)(a,b) := \C(a, Tb)$.

Given any functor $F\colon \A\to \C$, the right Kan extension $T_F=\Ran_FF$ is a monad on $\C$, the \emph{codensity monad} of $F$; hom-sets in the Kleisli category $\cate{Kl}(T_F)$ can be characterized as 
\[
\cate{Kl}(T_F)(c,c') \cong \int_a \Sets(\C(c', Fa), \C(c, Fa)).
\]
The proof is an exercise in coend-juggling, recalling that $T_F(\firstblank) \overset{(\refbf{kanend})}\cong \int_c Fc^{\C(\firstblank, Fc)}$.
\end{example}
\begin{example}\label{dualvecspace}
Let $V$ be a finite dimensional vector space over the field $K$; let $V^\lor$ denote the dual vector space of linear maps $V\to K$. Then there is a canonical isomorphism
\[
\int^V V^\lor \otimes V\cong K.
\]
The fastest way to see this is to notice that
\[
\int^V \hom(V,\firstblank)\otimes V\cong \Lan_{\id}(\id)\cong \id_{\cate{Vect}}
\]
(compare this argument with any proof trying to explicitly evaluate the coend!)
\end{example}
\begin{remark}
The universal cowedge $\hom(V,V)\xto{\alpha_V} K$ sends an endomorphism $f\colon V\to V$ to its \emph{trace} $\tau(f)\in K$ (which in this way acquires a universal property).

The above argument holds in fact in fair generality, adapting to the case where $V$ is an object of a compact closed monoidal category $\mathcal{C}$.
\end{remark}
\begin{exerciseset}
\begin{exercisepoints}
\item \label{closed.via.coends} Show that presheaf categories are cartesian closed, via coends: if $[\C^\opp,\Sets]$ is the category of presheaves on a small $\C$, then there exists an adjunction
\[
\Nat(P\times Q, R)\cong \Nat(P, R^Q)
\]
by showing that $R^Q(c) = \Nat(\yon_c\times Q, R)$ does the job (use the ninja Yoneda lemma, as well as \athm\refbf{naturalu}).
\item Use equations (\refbf{kanend}) and the ninja Yoneda lemma that $\Lan_\text{id}$ and $\Ran_\text{id}$ are the identity functors, as expected. Use again (\refbf{kanend}) and the ninja Yoneda lemma to complete the proof that $F\mapsto \Lan_F$ is a pseudofunctor, by showing that for $\A \xto{F}\B$, $\A\xto{G}\C\xto{H}\D$ there is a uniquely determined laxity cell for composition
\[
\Lan_H(\Lan_G(F))\cong \Lan_{HG}(F)
\]
(hint: coend-juggle with $\Lan_H(\Lan_G(F))d$ until you get $\int^{xy}(\D(Hx, d)\times \C(Gy, x))\cdot Fy$; now use the ninja Yoneda lemma plus co-continuity of the tensor, as suggested in Remark \refbf{cocotens}).
\item Prove in a similar way isomorphisms $(ii), (iii), (iv)$ in \athm\refbf{ninjayo} (hint: for $(ii)$ and $(iv)$ start from $\Sets\Big(y, \int_cHc^{\C(x,c)}\Big)$ and use again the end preservation property, cartesian closure of $\Sets$, and \athm\refbf{naturalu}).
\item Let $\C$ be a small compact closed monoidal category \cite{MR0470024}; Show that the functor $y\mapsto \int^x x^\lor \otimes y \otimes x$ carries the structure of a monad on $\C$.
\end{exercisepoints}
\end{exerciseset}