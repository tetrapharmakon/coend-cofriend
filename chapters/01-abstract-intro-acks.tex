\title[Ends]{This is the (co)end, my only (co)friend}
 
\author{Fosco Loregian}
\address{%
Department of Mathematics and Statistics             \newline
Masaryk University, Faculty of Sciences              \newline
Kotl\'{a}\v{r}sk\'{a} 2, 611 37 Brno, Czech Republic \newline
\href{mailto:loregianf@math.muni.cz}
   {\sf loregianf@math.muni.cz}
\href{mailto:loregianf@math.muni.cz}
   {\sf fosco.loregian@gmail.com}}

\begin{abstract}
The present note is a recollection of the most important and useful applications of co/end calculus. 
\emph{Co/ends} are particular universal objects in a category, defined for a functor $T\colon 
\C^\opp\times \C \to \D$. These objects behave in a similar way to \emph{co/limits} of a functor 
(they can easily be described as particular co/limits), but they come equipped with a richer, more 
flexible and descriptive \emph{calculus} (this word informally denotes a series of rules of manipulation 
that give theorems as formal consequences, similar to the way one can prove the RAPL theorem in 
classical category theory --the fact that \textbf{r}ight \textbf{a}djoints \textbf{p}reserve 
\textbf{l}imits); another playful informal term for this set of rules is \emph{co/end-fu}). We put a 
considerable effort in making arguments and constructions rather explicit and elementary, asking 
the reader only a minimal knowledge in basic category theory: after having given a series of 
preliminary definitions, we characterize co/ends as particular co/limits; then we derive a number 
of results directly from this characterization. The subsequent sections discuss the most interesting 
examples where co/end calculus serves as a powerful abstract way to do explicit computations in diverse 
fields like Algebra, Algebraic Topology and Category Theory as well as some generalizations to higher 
dimensional category theory. We mostly re\hyp{}enact a lot of classical results, inside and outside 
pure category theory, using co/end-fu. The appendices serve to sketch a number of results in 
theories heavily relying on co/end calculus; the reader who dares to arrive at this point will 
become a master in \begin{CJK*}{UTF8}{bsmi}端楔術\end{CJK*}, du\={a}nxi\={e} shù; literally 
``the art [of handling] terminal wedges'').
\end{abstract}

\subjclass[2010]{}
\keywords{
 end,
 coend, 
 dinatural transformation, 
 operad, 
 profunctor, 
 relator,
 Kan extension, 
 weighted limit, 
 nerve and realization, 
 promonoidal category, 
 Yoneda structure.}

\maketitle

{\small \tableofcontents}

\section*{Introduction.}
The purpose of the present survey is to familiarize its readers with the part of category theory that is called \emph{co/end calculus}, and aims to be a fairly complete account of its numerous application; there is now claim of originality in the ideas you're about to see: we put a special care in acknowledging carefully, where possible, each of the countless authors whose work was a source of inspiration in compiling this note. Among these, every erroneous or missing attribution must be ascribed to the mere ignorance of the author.

The original definition of a co/end as a universal object $\int T(c,c)$ for a functor $T\colon \C^\opp\times \C\to \D$ was given by N\@. Yoneda (who used an integral sign to denote them for the first time), and then refined by Mac Lane in his \cite{mac1970milgram} with applications to the `tensor product' of two functors, that generalizes the well-known description of $M_R\otimes {}_RN$ as a coequalizer for a right $R$\hyp{}module $M$ and a left $R$\hyp{}module $N$ on the same ring $R$. 

The introductory material appearing in section \refbf{section:due} is the most classical and comes almost \emph{verbatim} from the standard reference ``categories work'', \cite{McL}; the nerve\hyp{}realization formalism is a \emph{patchwork} of various results, scattered in the (algebraic) topology literature: few topology books mention the fact that the geometric realization of a simplicial set is a coend. The formal meaning of the construction is capable to destroy the complexity of many arguments that one is often forced to check by hand. We hope to fill this gap here.

The reformulation of operads using coends comes \emph{verbatim} from \cite{Kelly2005a}; the content of this paper was recently (April 2017) analyzed in an episode of the `Kan extension seminar \textsc{ii}', and posted on the $n$-café.

For what concerns the chapter discussing the theory of \emph{relators} (or \emph{profunctors}, or \emph{distributors} following J\@. Bénabou; our choice is instead supported by A\@. Joyal) the author immensely profited from some notes taken by T\@. Streicher, \cite{benabou2000distributors}; during May 2016 he had the pleasant opportunity to meet Thomas, and he fell in love with his mathematical style, near to the craftsmanship of certain watchmakers, but also vital and passionate. This note looks better after those pleasant days of friendly mathematics.

Chapter \refbf{section:weight} on weighted co/limits is taken almost verbatim from \cite[\textbf{II.7}]{riehl2014categorical}: the theory is however fairly classical, and became a milestone of enriched category theory. Emily's book is indisputably a complete and modern account of the theory of weighted co/limits: only a couple of implicit conceptual dependencies that appear as exercises there have been promoted here to explicit computations; the fact that we follow so faithfully her exposition must be interpreted not as an act of plagiarism, but instead ad an implicit invitation to get acquainted with such a wonderful book.

A subsequent section discusses co/end calculus in higher category theory, introducing lax co/ends in 2\hyp{}categories \cite{bozapalides1977finsgen,bozapalides1975fins,bozapalides1980some}, \emph{homotopy co/ends} \cite{DrorFar98,Isaacson} and \emph{$(\infty,1)$\hyp{}co/ends} in various models: first in Joyal\hyp{}Lurie's $\infty$\hyp{}categories (following \cite{gepner2015lax}), and then in simplicially enriched categories (mainly following \cite{cordier1997homotopy} and the general theory for enriched co/ends in \cite{dubuc1970kan,Graya}).

The relation between co/end calculus and homotopy co/limits is discussed, and a(n already) classical result \cite{Gamb} is presented to unify the two constructions classically given for the homotopy co/limit of a diagram; the compatibility of the co/end operation as a Quillen functor $\int\colon \Cat(\C^\opp\times\C,\D)\to \D$ is discussed when $\D$ is a sufficiently nice model category (our main reference to prove that $\int$ is a Quillen functor is \cite[\textbf{A.2.9.28}]{HTT}).

The last chapter about promonoidal categories comes from \cite{day1974embedding, street2012monoidal}; the subsequent appendices propose the reader to familiarize with the theory of promonoidal categories, and to show the initial results of \cite[\S \textbf{1}-\textbf{3}]{day2011monoidal}, a delightful and deep paper whose importance is far more than a source of unusual exercises.

Hence, the value --if any-- of this work lies, rather than in the originality of the discussion, in its strong will to be a simple and clear, yet exhaustive, account of the many reasons why every mathematician whose work has non\hyp{}empty intersection with category theory should know what co/ends are and how well they work.

Our aim is then threefold: the vast literature on category theory lacks a monographic account of this relatively big piece of basic theory; we hope the present document serves this purpose. Also, the inexperienced reader (either the beginner in the study of category theory, or the ones who exhausted the ``primary'' topics of their education) can find in this note a guide to familiarize with an extraordinarily valuable tool-set producing a large number of \emph{sterile}\footnote{We take the word sterile, from \cite{Leinster-cafe}, as ``\emph{Sterile} doesn't only mean infertile or unproductive. It's also what you want surgical instruments to be: clean, uncontaminated, disease-free.''} proofs, most of which are ``formally formal'' (although maybe not ``trivially trivial'') strings of natural isomorphisms. 

Finally, several people inside category theory underestimate the descriptive power of co/end calculus. It is then our firm opinion that this document could help even some experienced mathematicians, maybe working nearby category theory, that are never been exposed to this beautiful machinery. It is our hope that they could find their way to exploit co/ends as the fruitful tool they are.

There are few technical points to keep in mind to become a master in co/end-fu. After the first examples, the keen reader will certainly prefer to re\hyp{}write most of the proofs in the silence of their room, and we warmly invite them to do so; martial arts (or, if you refuse the analogy, integration rules) can be mastered only through endless (!) \emph{imitation} of a few basic techniques.% is for sure an unavoidable step in getting acquainted with the machinery of co/end calculus.%, and more generally with any machinery whatsoever. This is especially true for mathematical language, in the same sense we all learned integration rules by imitation and training reading precalculus books.

Maybe it's not a coincidence that the one you are about to see is \emph{another} integral calculus to be learned by means of examples and exercises. The analogy between coends, denoted as $\int T(c,c)$, \emph{could} be pushed further, but we refrain to do it, lest you think we are arrogantly claiming to be able to reduce the subtle art of integration to category theory. Nevertheless, we can't help but mention several insightful (formal and informal) analogies between mathematical analysis and co/end calculus: these are scattered throughout all the discussion, and we denote them with the special symbol $\itsnonsense$.\footnote{The author learned this funny notation during my freshman year, when he was handed \cite{de1996analisi} for the first time; the ``small\hyp{}eyes'' notation accompanied me throughout all my mathematical life until today. Various (facial) expressions advise different ways the reader is supposed to behave when she meets them. There are four such expressions: $\upeyes$, abstract material; $\downeyes$, standard exercises; $\righteyes$, material that you are supposed to meditate a lot; $\awful$, shattering exercises.} Whenever it appears, we advise the reader feeling uncomfortable with a certain dose of hand-waving and categorical juggling to raise their eyes and skip the paragraph.

% It has been said that ``Universal history is, perhaps, the history of a few metaphors'' \cite{Otras}: differential and integral calculus is undoubtedly one of such recurring themes. In some sense, the fortune of co/end calculus is based on the analogy which represents these universal objects by means of an integral symbol \cite{yoneda, day1969enriched, street2012monoidal}; this analogy is motivated by the Fubini theorem on the interchange of ``iterated integrals'':  unfortunately, despite its immense expressive and unifying power the language of coends seems to be woefully underestimated. No elementary book in category theory (apart a single chapter in the aforementioned \cite{McL}) seems to contain something more than a bare introduction of the basic elements of the language. 

% What's missing, in the humble opinion of the author, is an exhaustive and unitary source of examples, exercises and computations, showing their readers how co/end calculus can literally \emph{disintegrate} involved computations and reduce them to a bunch of canonical isomorphisms. 

% Trying to fill this gap in the literature has been the main motivation for the text you're about to read. It's up to you to decide it it is a clumsy attempt, or a partial success.

%\subsubsection*{\bf A final note, \today} It is impossible, at this point, to keep track of all the influences I received since when this project started. The present document doubled its length since its first version appeared on the arXiv. Probably someday a ``version 3'' will appear. Thanks everybody, especially to the people I acknowledge below.

\subsection*{Foundations, notation and conventions.} The main foundational convention we adopt throughout the paper is the assumption \cite{artin1972sga} that every set lies in a suitable Grothendieck universe. We implicitly fix such an universe $\mho$, whose elements are termed \emph{sets}; categories are hence always considered to be small with respect to \emph{some} universe: in particular we choose to adopt, whenever necessary, the so\hyp{}called \emph{two\hyp{}universe convention}, where we postulate the existence of a universe $\mho^+\ni \mho$ in which all the non-$\mho$\hyp{}small categories live. 

This rather common choice has nevertheless subtle consequences: as it is recorded in \cite{MR0396578,low2013universes} the existence and good behaviour of some co/limits and Kan extensions critically depends from the particular choice of a universe. We somehow follow a common tradition in avoiding any kind of reference to these problems, also because at least in some situations it is still possible to sweep this problem under sufficiently big a carpet, appealing some ``boundedness'' conditions keeping track of the cardinality of the involved constructions. There are in fact very few places where this remark could become a real issue (generally speaking, a problem arises anytime you want to perform a Kan extension along a functor defined over a non small category).

Several kinds of categorical structures (categories, and often also 2\hyp{}categories and bicategories, as well as instances of higher categories) will be denoted as boldface letters $\C,\D,\dots$; the context, or mere common sense, always clarifies for which value of $n$ we are doing $n$\hyp{}category theory. 

Functors between categories are denoted as capital Latin letters like $F,G,H,K$ and suchlike (although there can be little deviations to this rule, like for example in \S\refbf{sec:relators}); the category of functors $\C\to \D$ between two categories is denoted as $\textsf{Fun}(\C,\D)$, $\D^{\C}$, $[\C,\D]$ and suchlike; $\widehat{\C}$ is a shorthand for the category $[\C^\text{op},\Sets]$ of presheaves on $\C$; the canonical $\hom$\hyp{}bifunctor of a category $\C$ sending $(c,c')$ to the set of all arrows $\hom(c,c')\subseteq\hom(\C)$ is almost always denoted as $\C(\firstblank,\secondblank)\colon \C^\text{op}\times\C\to\cate{Sets}$, and the symbols $\firstblank$, $\secondblank$ are used as placeholders for the ``generic argument'' of a functor or bifunctor; morphisms in the category $\textsf{Fun}(\C,\D)$ (\ie natural transformations between functors) are often written in Greek, or Latin lowercase alphabet, and collected in the set $\Nat(F,G) = \D^{\C}(F,G)$. 

The simplex category $\bDelta$ is the \emph{topologist's delta} (opposed to the \emph{algebraist's delta} $\bDelta_+$ which has an additional initial object $[-1]\defequal \varnothing$), having objects \emph{nonempty} finite ordinals $[n]\defequal\{0<1\dots<n\}$; we denote $\Delta[n]$ the representable presheaf on $[n]\in\bDelta$, \ie the image of $[n]$ under the Yoneda embedding of $\bDelta$ in the category $\cate{sSet} = \widehat{\bDelta}$ of simplicial sets. More generally, we indicate the Yoneda embedding of a category $\C$ into its presheaf category with $\yon_\C$ --or simply $\yon$--, \ie with the hiragana symbol for ``yo''; this choice comes from \cite{Libland2015}, with which the author shares a similar aesthetics for peculiar notation. Whenever there is an adjunction $F\dashv G$ between functors, the arrow $Fa\to b$ in the codomain of $F$ and the corresponding arrow $a\to Gb$ in its domain are called \emph{mates} or \emph{adjuncts}; so, the notation ``the mate/adjunct of $f\colon Fa\to b$'' means ``the unique arrow $g\colon a\to Gb$ determined by $f$''. 

\subsubsection*{\bf Prerequisites}
We ask the reader for a minimum knowledge of basic category theory: in particular we feel free to assume that the reader is extremely fluent with all the basic definitions (category, functor, natural transformation, the slice category $\C/c$ of arrows $x\to c$ and its dual $c/\C$, comma categories and their basic properties, presheaves and their basic properties\dots) the essential features of co/limits, adjoint functors (especially the identities relating unit and counit of an adjunction $F\dashv G$), the validity and the consequences of the Yoneda lemma, and basic enriched category theory (the definition, and a few basic results on monoidal categories). The less elementary results and theorems are always sketched or the reader is precisely referred to pointers in the literature on the subject; comprehensive monographs on category theory always suffice to cover any missing spot we leave, and our choice of terminology is standard enough to facilitate the reader. Each section finishes with a set of exercises: some of them are easy, while others assume much more work, or the presence of a good book on the subject on your shelf.
\subsubsection*{\bf Acknowledgements}
In some sense, I am not the only author of this note, and for sure the less entitled to survey on whatever topic. I would like to thank {\sf T\@. Trimble}, {\sf E\@. Rivas} and {\sf A\@. Mazel\hyp{}Gee} for having read carefully the preliminary version of this document, suggesting improvements and corrections, having spotted a disgraceful number of errors, misprints and incoherent choices of notation. Their attentive proofreading has certainly increased the value --again, if there is any-- of the document you're about to read.

This humongous amount of errors did not prevent the first version of \emph{coend\hyp{}cofriend} to circulate, and being read, far more than I expected. And this is true to the point that at the moment of writing this is the paper with the highest number of (formal or moral) citations. I warmly thank the people who judged this clumsy list of examples a document worth to be read, improved and even studied.

Among the people who supported directly and indirectly the genesis of this paper: a conversation with {\sf A\@. Joyal} in a café in Paris, where I wrote the statement of Example \refbf{elts-as-coend} on a napkin to motivate the ubiquity and supremacy of coend\hyp{}fu, happened in June 2013 and convinced me to start the project; the \textsc{c}\&\textsc{p} colleagues and friends {\sf A\@. Gagna}, {\sf E\@. Lanari}, {\sf G\@. Mossa}, {\sf F\@. Genovese}, {\sf M\@. Vergura}, {\sf I\@. Di Liberti}, {\sf S\@. Ariotta}, {\sf G\@. Ronchi} endured sometimes unpleasant conversations on ``why every mathematician should know co/end calculus'';  {\sf D\@. Fiorenza}, a friend and an advisor I will never be able to refund for his constant support, spurred me to turn a series of chaotic sheets of paper into the present note; {\sf N\@. Gambino} offered me the opportunity to discuss the content of this note in front of his students in Leeds; in just a few days I realized years and years of meditation were still insufficient to \emph{teach} this subject. L\@. accompanied me there, making me sure that she's the best \emph{mate} for a much longer trip. 
\textsf{S\@.}%ofia
, \textsf{P\@.}%aolo
, \textsf{G\@.}%iovanni
, \textsf{C\@.} %aterina
opened me their doors when I was frail and broken\hyp{}hearted. Grazie.