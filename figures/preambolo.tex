% ########################################
% GENERAL PACKAGES
% \usepackage[usenames,dvipsnames]{xcolor}


\usepackage{
   amssymb
  ,amsfonts
  ,url
  ,amsmath
  ,etex
  ,stmaryrd
  ,lettrine
  ,tikz
  ,esint
  ,epigraph
  ,manfnt
  ,xspace
  ,etoolbox
  ,microtype
  ,lmodern
  ,marginnote
  ,rotating
  ,datetime
  ,mathtools
  ,hyphenat
  ,newclude}

\def\p{\text{pr}}

\usetikzlibrary{kD}
\usetikzlibrary{matrix,calc,fit,shapes,positioning}

\usepackage[spanish, greek, english]{babel}
\usepackage[utf8]{inputenc}
\usepackage[T1]{fontenc}

\usepackage[all,2cell,cmtip]{xy}\UseAllTwocells
% ########################################

% ########################################
% STYLING
\setlength{\epigraphwidth}{0.65\textwidth}

\makeatletter
  \def\@cite#1#2{[\textbf{#1}\if@tempswa , #2\fi]}
  \def\@biblabel#1{[\textsf{#1}]}
\makeatother

\newlength{\seplen}
\setlength{\seplen}{5pt}
%
\newlength{\sepwid}
\setlength{\sepwid}{.4pt}
%
\def\firstblank{\,\rule{\seplen}{\sepwid}\,}
\def\secondblank{\firstblank\llap{\raisebox{2pt}{\firstblank}}}

\renewcommand{\textbf}[1]{\text{\fontseries{b}\selectfont{\upshape #1}}}
  \newcommand{\trans}[1]{\textcolor{red}{\bf #1}}
  \newcommand{\refbf}[1]{\textbf{\ref{#1}}}

\usepackage{enumitem}
\setlist[itemize]{noitemsep, topsep=2pt}
\setlist[enumerate]{label=(\oldstylenums{\arabic*}), noitemsep, topsep=2pt}


\usepackage[cal=boondoxo]{mathalfa}
% ########################################

% ########################################
% COMMANDS FOR TEXT
\def\mho{\rotatebox[origin=c]{180}{$\Omega$}}

\def\proP{\textsc{p}}
\def\proQ{\textsc{q}}
\def\proH{\textsc{h}}
\def\proL{\textsc{l}}
\def\proK{\textsc{k}}
\def\proG{\textsc{g}}

\providecommand{\abbrv}[1]{#1.\@\xspace}
  \providecommand{\ie}{\abbrv{i.e}}
  \providecommand{\etc}{\abbrv{etc}}
  \providecommand{\prof}{\abbrv{prof}}
  \providecommand{\viz}{\abbrv{viz}}
  \providecommand{\eg}{\abbrv{e.g}}
  \providecommand{\achap}{\abbrv{Ch}}
  \providecommand{\adef}{\abbrv{Def}}
  \providecommand{\aprop}{\abbrv{Prop}}
  \providecommand{\athm}{\abbrv{Thm}}

\usepackage{CJKutf8}

% The yoneda embedding: the letter "yo" in hiragana
\newcommand{\yon}{\text{\begin{CJK}{UTF8}{min}よ\end{CJK}}}

\newcommand{\wk}{\textsc{wk}}
\newcommand{\cof}{\textsc{cof}}
\newcommand{\fib}{\textsc{fib}}

\def\itsnonsense{\upeyes}

\setcounter{tocdepth}{1} 
\usepackage[titletoc]{appendix}
% \usepackage[
%        a4paper
%       ,bottom=6cm
%       ,left=4.2cm
%       ,right=4.2cm
%       ,top=4.5cm]{geometry}
% ########################################

% ########################################
% COMMANDS FOR MATH

\usepackage{amsthm}

\newtheoremstyle{reference}
   {}                
   {}                
   {}              
   {}                      
   {\fontseries{b}\selectfont}              
   {:}                     
   {.2em}                  
   {\thmname{#1}           
    \thmnumber{#2}         
    \thmnote{{\sc [#3]}}} 

\theoremstyle{reference}
  \newtheorem{theorem}{Theorem}[section]
  \newtheorem{lemma}[theorem]{Lemma}
  \newtheorem{proposition}[theorem]{Proposition}
  \newtheorem{example}[theorem]{Example}
  \newtheorem{exercise}[theorem]{Exercise}
  \newtheorem{remark}[theorem]{Remark}
  \newtheorem{definition}[theorem]{Definition}
  \newtheorem{corollary}[theorem]{Corollary}
  \newtheorem{notat}[theorem]{Notation}
  \newtheorem*{acknowledgements}{Acknowledgements}
  \newtheorem{scholium}[theorem]{Scholium}
  \newtheorem{counterex}[theorem]{Counterexample}
  % starred
  \newtheorem*{theorem*}{Theorem}
  \newtheorem*{lemma*}{Lemma}
  \newtheorem*{proposition*}{Proposition}
  \newtheorem*{example*}{Example}
  \newtheorem*{exercise*}{Exercise}
  \newtheorem*{remark*}{Remark}
  \newtheorem*{definition*}{Definition}
  \newtheorem*{corollary*}{Corollary}
  \newtheorem*{notat*}{Notation}
  \newtheorem*{scholium*}{Scholium}
  \newtheorem*{counterex*}{Counterexample}

\newcommand{\bDelta}{\pmb\Delta}
\newcommand{\Dist}{\cate{Relt}}
\newcommand{\varsSet}{\widehat{\bDelta}}
\newcommand{\simplex}[1]{{\Delta[#1]}}
\newcommand{\xto}[2][]{\xrightarrow[#1]{#2}}
\newcommand{\xot}[2][]{\xleftarrow[#1]{#2}}
\newcommand{\bsmat}{\left[\begin{smallmatrix}}
\newcommand{\esmat}{\end{smallmatrix}\right]}
\renewcommand{\phi}{\varphi}
\newcommand{\var}[3][]{
  \left[\begin{smallmatrix} #2 \\ 
  #1\downarrow \\ #3 
  \end{smallmatrix}\right]}
\newcommand{\laxto}{\dashrightarrow}
\newcommand{\ccirc}{\diamond}

\DeclareMathOperator{\id}{id}
\DeclareMathOperator{\Lan}{Lan}
\DeclareMathOperator{\Ran}{Ran}
\DeclareMathOperator{\Rift}{Rift}
\DeclareMathOperator{\Lift}{Lift}
\DeclareMathOperator{\ob}{Ob}
\newcommand{\opp}{\text{op}}
\newcommand{\oo}{\text{oo}}
\newcommand{\Ex}{\text{Ex}}

\newcommand{\dashlim}{\text{-lim }}
\providecommand{\defequal}  {\overset{\triangle}{=}}

\newcommand{\cate}[1]{\text{\fontseries{b}\selectfont{#1}}}
  \newcommand{\A}     {\cate{A}}
  \newcommand{\B}     {\cate{B}}
  \newcommand{\C}     {\cate{C}}
  \newcommand{\Cat}   {\cate{Cat}}
  \newcommand{\Gpd}   {\cate{Gpd}}
  \newcommand{\D}     {\cate{D}}
  \newcommand{\K}     {\cate{K}}
  \renewcommand{\P}   {\cate{P}}
  \newcommand{\catS}  {\cate{S}}
  \newcommand{\Sets}  {\cate{Sets}}
  \newcommand{\Set}   {\cate{Sets}}
  \newcommand{\sSet}  {\cate{sSet}}

  \newcommand{\N}{\mathbb{N}}
  \newcommand{\Z}{\mathbb{Z}}

  \newcommand{\Top}   {\cate{Top}}
  \newcommand{\E}     {\mathcal{E}}
  \renewcommand{\L}   {\mathcal{L}}
  \newcommand{\M}     {\mathcal{M}}
  \newcommand{\R}     {\mathcal{R}}
  \newcommand{\V}     {\mathcal{V}}
  \newcommand{\tee}     {\mathfrak{t}}
  \newcommand{\T}     {\mathfrak{T}}
  \newcommand{\VCat}  {\V\text{-}\Cat}

\def\Nat{\textsf{Nat}}
\def\Fun{\textsf{Fun}}
\def\LNat{\textsf{LNat}}
\def\bemo{\flat}
\def\diesis{\sharp}
\newcommand{\pto}{\leadsto}
\newcommand{\wlim}[1]{{\textstyle \varprojlim^{#1}}}
\newcommand{\wcolim}[1]{{\textstyle \varinjlim^{#1}}}
\newcommand{\hocolim}{\underrightarrow{\text{holim}}\,}
\newcommand{\holim}{\underleftarrow{\text{holim}}\,}
\newcommand{\elts}[2]{#1\!\int\! #2}
\newcommand{\tw}{\textsc{tw}}

\newcommand{\twoint}{\sqint}
% {\mathchoice%
% {\ooalign{$\infty$\cr\hfil$\displaystyle\int$\hfil}}%display
% {\ooalign{$\scriptstyle\infty$\cr\hfil$\int$\hfil}}%text
% {\ooalign{$\scriptscriptstyle\infty$\cr\hfil$\scriptstyle\int$\hfil}}%script
% {\ooalign{$\scriptscriptstyle\infty$\cr\hfil$\scriptscriptstyle\int$\hfil}}}%script^2
\newcommand{\infint}{\oint}

% ########################################

% ########################################
% ALTRO
% DECOMMENTA X HYPHENS
% \tracingparagraphs=1
% \tracingonline=1
% \input{ipenate}

\newcommand{\arXivPreprint}[1]{arXiv preprint \href{http://arxiv.org/abs/#1}{arXiv:#1}}


\newcommand{\standard}{
\draw[ultra thin] (0.25,0) -- (1.75,0) -- (1.75,1) -- (0.25,1) -- cycle;
\draw (.5,0) -- (.5,1);
\draw (1,0) .. controls (1,.5) and (1.5,.5) .. (1.5,0);
\draw (1,1) .. controls  (1,.5) and (1.5,.5) .. (1.5,1);
}

\newcommand{\standardbis}[3]{
\draw[ultra thin] (0.25,0) -- (1.75,0) -- (1.75,1) -- (0.25,1) -- cycle;
\draw[#1] (.5,0) -- (.5,1);
\draw[#2] (1,0) .. controls (1,.5) and (1.5,.5) .. (1.5,0);
\draw[#3] (1,1) .. controls  (1,.5) and (1.5,.5) .. (1.5,1);
}

\newcommand{\first}[1]{\draw[ultra thin] (0.25,0) -- (1.75,0) -- (1.75,1) -- (0.25,1) -- cycle;
\draw (.5,0) -- (.5,1);
\draw[xshift=.5cm] (.5,0)  -- (.5,1);
\draw[xshift=1cm] (.5,0)-- (.5,1);
\filldraw[lightgray!70] (.5,.5) circle (6pt) node[black] {$ #1 $};}
\newcommand{\second}[1]{\draw[ultra thin] (0.25,0) -- (1.75,0) -- (1.75,1) -- (0.25,1) -- cycle;
\draw (.5,0) -- (.5,1);
\draw[xshift=.5cm] (.5,0)  -- (.5,1);
\draw[xshift=1cm] (.5,0)-- (.5,1);
\filldraw[lightgray!70] (1,.5) circle (6pt) node[black] {$ #1 $};}
\newcommand{\third}[1]{\draw[ultra thin] (0.25,0) -- (1.75,0) -- (1.75,1) -- (0.25,1) -- cycle;
\draw (.5,0) -- (.5,1);
\draw[xshift=.5cm] (.5,0)  -- (.5,1);
\draw[xshift=1cm] (.5,0)-- (.5,1);
\filldraw[lightgray!70] (1.5,.5) circle (6pt) node[black] {$ #1 $};}


%%%%%%%%%%%%%%%%%%

\newcommand{\upeyes}{%
  \begin{tikzpicture}
  \draw circle (3pt);
  \draw[fill,yshift=1.5pt] circle (1.5pt);
  \begin{scope}[xshift=7pt]
  \draw circle (3pt);
  \draw[fill,yshift=1.5pt] circle (1.5pt);
  \end{scope}
  \end{tikzpicture}%
}

\newcommand{\downeyes}{%
  \begin{tikzpicture}
  \draw circle (3pt);
  \draw[fill,yshift=-1.5pt] circle (1.5pt);
  \begin{scope}[xshift=7pt]
  \draw circle (3pt);
  \draw[fill,yshift=-1.5pt] circle (1.5pt);
  \end{scope}
  \end{tikzpicture}%
}

\newcommand{\righteyes}{%
  \begin{tikzpicture}
  \draw circle (3pt);
  \draw[fill,xshift=1.5pt] circle (1.5pt);
  \begin{scope}[xshift=7pt]
  \draw circle (3pt);
  \draw[fill,xshift=1.5pt] circle (1.5pt);
  \end{scope}
  \end{tikzpicture}%
}

\newcommand{\awful}{%
   \begin{tikzpicture}
   \draw circle (3pt);
   \draw[fill] circle (1.5pt);
   \begin{scope}[xshift=7pt]
   \draw circle (3pt);
   \draw[fill] circle (1.5pt);
   \end{scope}
   \end{tikzpicture}%
}


% ########################################

% ########################################
% ALCUNE COSE VANNO SEMPRE IN FONDO

\usepackage{hyperref}
\hypersetup{%
  pdftoolbar=   true,
  pdfmenubar=   true,
  pdffitwindow= true,
  pdftitle=     {This is the co/end},
  pdfauthor=    {F. Loregian},
  colorlinks=   true,
  linkcolor=    black,
  citecolor=    Magenta}

% ########################################

\def\[{\begin{equation}}
\def\]{\end{equation}}

\usepackage[framemethod=TikZ]{mdframed}

\makeatletter
\def\exerciseboxtag{
  \useasboundingbox (P) rectangle (P);
  \node at (P-|O) [
      draw=black!70,
      line width=.4pt,
      fill=black!2,
      rectangle,
      rounded corners=2pt,
      inner sep=.3em,
      anchor=west,
      xshift=3em,
      font=\mdf@frametitlefont,
      execute at begin node=\strut~,
      execute at end node=~
  ] {Exercises for \S\thesection};
}
\makeatother

\mdfdefinestyle{exstyle}{%
  % these two account for spacing around the tag:
  skipabove=1.6em, innertopmargin=1.6em,
  middlelinewidth=.4pt,
  roundcorner=1.6pt,
  linecolor=black!80,
  backgroundcolor=black!2,
  frametitlefont=\bfseries,
  % settings={\global\stepcounter{question}},
  singleextra={\exerciseboxtag},
  firstextra={\exerciseboxtag}
}

\newenvironment{exerciseset}
  {\begin{mdframed}[style=exstyle]\footnotesize}
  {\end{mdframed}}

\allowdisplaybreaks

\usepackage{enumitem}
\newlist{exercisepoints}{enumerate}{1}
\setlist[exercisepoints]{
  label={\sffamily\mdseries E\arabic*.},
  leftmargin=2em,
  % before=\raggedright
}



% \tikzstyle{title} =
%     [draw=lightgray, thick, fill=lightgray,% 
%         text=black, rectangle,  
%         right, minimum height=.4cm]

% \newcommand*\theex{Exercises for \S\thesection}
% \makeatletter
% \mdfdefinestyle{exstyle}{%
%     leftmargin=1em,%
%     rightmargin=1em,%
%     middlelinewidth=1.2pt,%
%     roundcorner=5pt,%
%     linecolor=lightgray,%
%     backgroundcolor=lightgray!20,%
%     innertopmargin=1.2\baselineskip,%
%     skipabove=2em,%
%     skipbelow=1em,%
%     needspace=1em,%
%     frametitlefont=\sffamily\bfseries,%
%     %settings={\global\stepcounter{question}},%
%     singleextra={%
%         \useasboundingbox (P) rectangle (P);
%         \node[title,xshift=1cm] at (P-|O) %
%             {\mdf@frametitlefont{\theex}~};},%
%     firstextra={%
%         \useasboundingbox (P) rectangle (P);
%             \node[title,xshift=1cm] at (P-|O) %
%                     {~\mdf@frametitlefont{\theex}~};},
% }
% \makeatother
% \newenvironment{exerciseset}%
% {\begin{mdframed}[style=exstyle]\footnotesize}%
% {\end{mdframed}}
