%------------------------------------------------------------------------------
% Beginning of example.tex
%------------------------------------------------------------------------------
%
% LaTeX sample file for Rendiconti del Seminario Matematico della Universit\`a di Padova
%
\documentclass{RSMUP}

\address[e-mail address]{Author's name, Department, University, PO Box or Street, City, Country}


\address[e-mail address]{Author's name, Department, University, PO Box or Street, City, Country}

%\address[e-mail address]{Author's name, Department, University, PO Box or Street, City, Country}

%\address[e-mail address]{Author's name, Department, University, PO Box or Street, City, Country}


%\received[revision date]{submission date}

% Uncomment the next line if you would like your equations to be numbered according to sections:
%\numberwithin{equation}{section}
%\numberwithin{figure}{section}
%\numberwithin{table}{section}


\newtheorem{theorem}{Theorem}[section]
\newtheorem{corollary}[theorem]{Corollary}
\newtheorem{lemma}[theorem]{Lemma}
\newtheorem{proposition}[theorem]{Proposition}
\newtheorem{conjecture}[theorem]{Conjecture}

\newtheorem*{coro}{Corollary} %%%% for unnumbered statements

\theoremstyle{definition}
\newtheorem{definition}[theorem]{Definition}
\newtheorem{example}[theorem]{Example}
\newtheorem{remark}[theorem]{Remark}




%% Create a new command named "\eps" as abbreviation for \varepsilon

\newcommand{\eps}{\varepsilon}	 % Epsilon



% -- New Operators --
%Use the command \DeclareMathOperator to define a new log-like function in LaTeX;
%see Section 2.4 below


\DeclareMathOperator{\Hom}{Hom}				
\DeclareMathOperator{\Ker}{Ker}
\DeclareMathOperator{\tr}{tr}
\DeclareMathOperator{\sgn}{sgn}


\title[Title of paper]{Title of paper}

\author[First author's name \ --\  Second author's name]{First author's name\thanks{The author is grateful to
the Max Planck Institute (Bonn) for hospitality during the writing of
this paper.} \ --\ Second author's name\thanks{The author is grateful to
IHES (Bures-sur-Yvette) for hospitality during the writing of
this paper.}}



\begin{document}


\maketitle



\begin{abstract}
This file explains how to prepare a contribution for publication in
\textit{Rendiconti del Seminario Matematico della Universit\`a di Padova}.
\end{abstract}

\begin{classification}
11M32; 14F20,  19F27.
\end{classification}

\begin{keywords}
$L$-function,  Selmer group.
\end{keywords}






\section{Introduction}
Authors are requested to use  standard \LaTeX\ and the class file
\begin{verbatim}
RSMUP.cls
\end{verbatim}
This style file is very similar to the standard article style file, and it loads amsmath,
amsfonts, amssymb, latexsym, and with amsthm.sty included. It sets the page size to
\begin{verbatim}
\textheight=192mm
\textwidth=125mm
\end{verbatim}
so you should not change the page size. We suggest you use this sample TeX file as a model,
modifying it where appropriate.

The \TeX\ source file should begin with
\begin{verbatim}
\documentclass{RSMUP}
\end{verbatim}
Enter the name(s) of the author(s) using the tag
\begin{verbatim}
\address[e-mail address]{Author's address}
\end{verbatim}
Each author's name should be entered with a separate \verb+\+{\tt address} command.
No personal style files should be used. Each paper should contain the 2000 Mathematics
Subject Classification.  Please avoid one-letter lower case newly defined commands like
\begin{verbatim}
\def\e{\varepsilon} or \newcommand{\e}{\varepsilon}
\end{verbatim}
since this can interfere with conversion of your article to Times fonts later. Use instead something
like:
\begin{verbatim}
\newcommand{\eps}{\varepsilon}
\end{verbatim}

\section{Some rules}

In order to achieve a uniform appearance of all the contributions, we encourage
you to observe the following rules when preparing your article.

\subsection{Section and subsections} Sections and paragraphs are obtained using the commands
\begin{verbatim}\section{title of section} \subsection{...} \subsubsection{...}\end{verbatim}
and unnumbered sections and paragraphs are obtained using their starred forms:
\begin{verbatim}\section*{title of section} \subsection*{...} \subsubsection*{...}\end{verbatim}


\subsection{Displayed formulas} If you have displayed formulas consisting of more than one line
we recommend to you use
\begin{verbatim}\begin{align}...\end{align}\end{verbatim}
 instead of
\begin{verbatim}\begin{eqnarray}...\end{eqnarray}\end{verbatim}
(respectively the  starred forms) since the former yields a better spacing. Compare:
\begin{align}
A&=  f(x_i)= F'(x),
\\
B&=  g(x_i)= G'(x),
\end{align}
\begin{eqnarray}
A&=& f(x_i)= F'(x),
\\
B&=& g(x_i)= G'(x).
\end{eqnarray}
In case you do not want the numbering for every line, type
\begin{verbatim}\nonumber\end{verbatim}
at the end of the line where you do not want a number.
\begin{align}
A&=  f(x_i)= F'(x),\nonumber
\\
B&=  g(x_i)= G'(x).
\end{align}
If you want a number for the complete block, this works:
\begin{verbatim}
\begin{equation}\begin{split}...\end{split}\end{equation}
\end{verbatim}
\begin{equation}
\begin{split}
A&=  f(x_i)= F'(x),
\\
B&=  g(x_i)= G'(x).
\end{split}
\end{equation}
If you prefer to number equations in the form (2.1), (2.2), \dots, add the line
\begin{verbatim}
\numberwithin{equation}{section}
\end{verbatim}
to the preamble of your document.


\subsection{Theorems and alike}

For theorems, lemmas, definitions, etc.\ use the standard syntax.

\begin{verbatim}
\begin{theorem}...\end{theorem}, \begin{lemma}...\end{lemma}, etc.
\end{verbatim}

Put optional arguments into square brackets (``Theorem, \cite{Jon}'' in the example below).

\begin{theorem}[Theorem 13.14, \cite{Jon}]
Let $L$  be an oriented link and let $\alpha \in B_{2m}$ be such that
$\tilde{\alpha}=L$ as unoriented links. Then there is a $k\in \mathbb{R}$, $2k \in \mathbb{Z}$,
with $V_L(t)= t^k(-(t+1))^{m-1} \phi(\pi_0(\alpha))$.
\end{theorem}

\begin{definition}
A \emph{preference order}  (or \emph{preference relation}) on $\mathcal X$ is a binary
relation $\succ$ with the following two properties.
\begin{enumerate}

\item \emph{Asymmetry:} If $x\succ y$, then $y\not\succ x$.

\item \emph{Negative transitivity}: If $x\succ y$ and $z\in{\mathcal X}$, then
          either $x\succ z$ or $z\succ y$ or both must hold.

\end{enumerate}
\end{definition}

In this example file, enumerations of theorems,
lemmas definitions, etc.\ appear consecutively.
If you want separate numbering (Theorem 2.1, Definition 2.1) change e.g.\
\begin{verbatim}
\newtheorem[theorem]{definition}
\end{verbatim}
to
\begin{verbatim}
\newtheorem{definition}{Definition}[section]
\end{verbatim}

If you want a statement unnumbered, just define
\begin{verbatim}
\newtheorem*{coro}{Corollary}
\end{verbatim}
to obtain

\begin{coro}
If $L$ and $L'$ are two oriented links which are isotopic as unoriented links,
then there is a $k\in \mathbb{Z}$ such that
\[
V_L(t)= t^k V_{L'}(t).
\]
\end{coro}

For a proof, use

\begin{verbatim}
\begin{proof}...\end{proof}
\end{verbatim}

An end-of-proof sign $\Box$ is set automatically.

\begin{proof}
This finishes the proof of the corollary.
\end{proof}

You can also make remarks and give examples with the commands
\begin{verbatim}
\begin{remark}...\end{remark}
\begin{example}...\end{example}
\end{verbatim}
which will produce:

\begin{remark}
This is an example of a `remark' element.
\end{remark}

\begin{example}
This is an example of an `example' element.
\end{example}


\subsection{Operator names}

There are several \TeX-commands setting things automatically upright like $\det$, $\sin$,\dots\,.
If you need operators not predefined, simply define e.g.
\begin{verbatim}
\DeclareMathOperator{\Hom}{Hom}
\DeclareMathOperator{\Ker}{Ker}
\end{verbatim}
and then use
\begin{verbatim}
\Hom, \Ker
\end{verbatim}
to obtain
$$
\varphi \in \Hom(G/H) \Longrightarrow \Ker(\varphi) \not= \{0\}.
$$
It is accepted typographical standard that abbreviated mathematical expressions standing for ``words''
appear in roman (upright) typeface.


\section{Lists}

\subsection{Numbered lists}
For numbered lists, you should use the \LaTeX\ command
\begin{verbatim}
\begin{enumerate}
\item First item
\item Second item
\end{enumerate}
\end{verbatim}
in a nested form, and this will produce:

\begin{enumerate}
\item First item.
\item Second item.
\begin{enumerate}
\item First subitem.
\item Second subitem.
\begin{enumerate}
\item First subsubitem.
\item Second subsubitem.
\end{enumerate}
\item Third subitem.
\end{enumerate}
\item Third item.
\end{enumerate}

\subsection{Bulleted lists}
For a bulleted list, you can use the command
\begin{verbatim}
\begin{itemize}
\item First item
\item Second item
\end{itemize}
\end{verbatim}
which will produce:

\begin{itemize}
\item First item
\item Second item
\item Third item
\end{itemize}


\section{References}

Citations should always be made with the \TeX\ command
\begin{verbatim}\cite{}\end{verbatim}
Also, when citing several works at the same time, you should use
\begin{verbatim}\cite{paper1}, \cite{paper2}, \cite{paper3}\end{verbatim}
as, for example, in \cite{And}, \cite{FrQu}, \cite{Jon}.

It follows a list of references showing you the style in which books and journal articles
should be listed.

\frenchspacing
\begin{thebibliography}{7}

\bibitem{And}   S. Bloch -- K. Kato,
\textit{$L$-functions and Tamagawa numbers of motives},
 in:  \textit{The Grothendieck Festschrift}, Vol. I,
 Progr. Math. 86, Birkh\"auser, Boston 1990, P. Cartier, et al., eds.,
pp. 333--400.

\bibitem{FrQu}
 J. S. Milne, \textit{Etale cohomology},
Princeton University Press, 1980.

\bibitem{Jon} F. Cafiero,  \textit{Sui problemi ai limiti relativi ad un'equazione differenziale ordinaria del primo ordine e dipendente da un parametro},
Rend. Sem. Mat. Univ. Padova,    \textbf{18} (1949), pp. 239--257.


\bibitem{Luk} M. A. Seveso, \textit{Stark--Hegner points and Selmer groups of abelian varieties},  PhD thesis,
University of Milan, Federigo Enriques Department of Mathematics, 2009.

\end{thebibliography}

\end{document} 