\usepackage[cal=boondoxo]{mathalfa}


\setlength{\epigraphwidth}{0.65\textwidth}

\makeatletter
  \def\@cite#1#2{[\textbf{#1}\if@tempswa , #2\fi]}
  \def\@biblabel#1{[\textsf{#1}]}
\makeatother

\newlength{\seplen}
\setlength{\seplen}{5pt}
%
\newlength{\sepwid}
\setlength{\sepwid}{.4pt}
%
\def\firstblank{\,\rule{\seplen}{\sepwid}\,}
\def\secondblank{\firstblank\llap{\raisebox{2pt}{\firstblank}}}

\renewcommand{\textbf}[1]{\text{\fontseries{b}\selectfont{\upshape #1}}}
  \newcommand{\trans}[1]{\textcolor{red}{\bf #1}}
  \newcommand{\refbf}[1]{\textbf{\ref{#1}}}

\usepackage{enumitem}
\setlist[itemize]{noitemsep, topsep=2pt}
\setlist[enumerate]{label=(\oldstylenums{\arabic*}), noitemsep, topsep=2pt}

\setcounter{tocdepth}{1} 
\usepackage[titletoc]{appendix}
\usepackage[
       a4paper
      ,bottom=6cm
      ,left=4.2cm
      ,right=4.2cm
      ,top=4.5cm]{geometry}

\usepackage{amsthm}

\newtheoremstyle{reference}
   {}                
   {}                
   {}              
   {}                      
   {\fontseries{b}\selectfont}              
   {:}                     
   {.2em}                  
   {\thmname{#1}           
    \thmnumber{#2}         
    \thmnote{{\sc [#3]}}} 

\theoremstyle{reference}
  \newtheorem{theorem}{Theorem}[section]
  \newtheorem{lemma}[theorem]{Lemma}
  \newtheorem{proposition}[theorem]{Proposition}
  \newtheorem{example}[theorem]{Example}
  \newtheorem{exercise}[theorem]{Exercise}
  \newtheorem{remark}[theorem]{Remark}
  \newtheorem{definition}[theorem]{Definition}
  \newtheorem{corollary}[theorem]{Corollary}
  \newtheorem{notat}[theorem]{Notation}
  \newtheorem*{acknowledgements}{Acknowledgements}
  \newtheorem{scholium}[theorem]{Scholium}
  \newtheorem{counterex}[theorem]{Counterexample}
  % starred
  \newtheorem*{theorem*}{Theorem}
  \newtheorem*{lemma*}{Lemma}
  \newtheorem*{proposition*}{Proposition}
  \newtheorem*{example*}{Example}
  \newtheorem*{exercise*}{Exercise}
  \newtheorem*{remark*}{Remark}
  \newtheorem*{definition*}{Definition}
  \newtheorem*{corollary*}{Corollary}
  \newtheorem*{notat*}{Notation}
  \newtheorem*{scholium*}{Scholium}
  \newtheorem*{counterex*}{Counterexample}

\usepackage[framemethod=TikZ]{mdframed}

\makeatletter
\def\exerciseboxtag{
  \useasboundingbox (P) rectangle (P);
  \node at (P-|O) [
      draw=black!70,
      line width=.4pt,
      fill=black!2,
      rectangle,
      rounded corners=2pt,
      inner sep=.3em,
      anchor=west,
      xshift=3em,
      font=\mdf@frametitlefont,
      execute at begin node=\strut~,
      execute at end node=~
  ] {Exercises for \S\thesection};
}
\makeatother

\mdfdefinestyle{exstyle}{%
  % these two account for spacing around the tag:
  skipabove=1.6em, innertopmargin=1.6em,
  middlelinewidth=.4pt,
  roundcorner=1.6pt,
  linecolor=black!80,
  backgroundcolor=black!2,
  frametitlefont=\bfseries,
  % settings={\global\stepcounter{question}},
  singleextra={\exerciseboxtag},
  firstextra={\exerciseboxtag}
}

\newenvironment{exerciseset}
  {\begin{mdframed}[style=exstyle]\footnotesize}
  {\end{mdframed}}

\allowdisplaybreaks

\newlist{exercisepoints}{enumerate}{1}
\setlist[exercisepoints]{
  label={\textbf{E\arabic*}},
  leftmargin=2em,
  % before=\raggedright
}